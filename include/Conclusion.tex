\section{Conclusion}

Pour ce premier rapport concernant le projet \textbf{Réception LIN} de conception de circuit numérique, 
nous avons suivi plusieurs étapes successives afin de concevoir ce système.  

Dans un premier temps, lors des séances de TD, nous avons décomposé notre étude en plusieurs parties afin de répondre au cahier des charges. 
L’utilisation du \textbf{diagramme en Y} nous a permis d’analyser séparément chacune de ces étapes et de structurer notre démarche. 
\newline

La phase de \textbf{spécification fonctionnelle} nous a permis de définir l’ensemble des ressources 
nécessaires, 
en lien direct avec le cahier des charges, ainsi que le nombre de blocs et les signaux de base.  
\newline

Ensuite, la \textbf{solution architecturale} a permis de préciser chaque partie en les reliant à des 
modèles connus 
(machines séquentielles, machines de Moore ou machines de Mealy). 
Cette étape a été essentielle pour découper notre système en une partie opérative et une partie commande :  
\newline

\begin{itemize}
    \item la partie opérative a été conçue à partir de blocs logiques simples (multiplexeurs, bascules D, etc.) ;  
    \item la partie commande a été décrite sous forme d’automates, associés à des machines de Moore ou de Mealy, 
    afin d’obtenir une description claire et structurée.  
\end{itemize}
Cette méthodologie nous a permis d’écrire un code plus simple, lisible et cohérent.  

La phase de \textbf{simulation} a ensuite validé le fonctionnement du système en stimulant les différents ports d’entrée 
et en observant les sorties.  
\newline

La phase de \textbf{synthèse} nous a permis de vérifier la logique interne du système à travers les schémas RTL générés. 
Ces schémas ont ensuite été exploités dans Vivado, qui a associé les différentes fonctions logiques aux \textbf{LUT}.  
\newline

Enfin, l’\textbf{implémentation} nous a donné accès au routage sur FPGA. 
Cette étape a permis de vérifier concrètement l’affectation des ressources matérielles et le câblage interne du système.  
\newline

En conclusion, ce premier travail nous a permis de valider l’\textbf{interface microprocesseur}. 
La suite du projet consistera à finaliser la conception complète du récepteur LIN afin de répondre à l’intégralité du cahier des charges.
