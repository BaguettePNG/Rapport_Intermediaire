\documentclass{article}
\usepackage[utf8]{inputenc}   % Pour les fichiers en UTF-8
\usepackage[french]{babel}
\usepackage{float}
\usepackage{xcolor}
\usepackage{listings}
\usepackage{lipsum} 
\usepackage{eso-pic}
\usepackage{tikz}
\usepackage{everypage}
\usepackage{subcaption}
\usepackage{fancyhdr}
\usepackage[absolute,overlay]{textpos}
\usepackage{fancyheadings}
\usepackage{lastpage}
\usepackage{eso-pic}
\usepackage[french]{babel}
\usepackage[hidelinks]{hyperref}
\usepackage{float}
\usepackage[utf8]{inputenc}
\usepackage{titlesec}
\usepackage{amsmath}
\usepackage{amssymb}
\usepackage{pifont}
\usepackage{pgfplots}
\pgfplotsset{compat=1.18}
\usepackage{amsmath}
\usepackage{amssymb}
\usepackage{siunitx}
\usepackage{tabularx} % à mettre dans le préambule
\usepackage{lastpage} % pour \pageref{LastPage}
\usepackage{hyperref}

% Définition du style VHDL
\lstdefinestyle{VHDLStyle}{
    language=VHDL,
    basicstyle=\ttfamily\small,      % police monospaced et taille petite
    keywordstyle=\color{blue}\bfseries, % mots-clés en bleu et gras
    commentstyle=\color{green!60!black}\itshape, % commentaires en vert italique
    stringstyle=\color{red},         % chaînes de caractères en rouge
    numbers=left,                    % numéros de lignes à gauche
    numberstyle=\tiny\color{gray},   % style des numéros
    stepnumber=1,                    
    numbersep=5pt,                   
    backgroundcolor=\color{gray!10}, % couleur de fond du code
    frame=single,                    % encadré autour du code
    rulecolor=\color{black},         
    breaklines=true,                 % retour à la ligne automatique
    breakatwhitespace=true,          
    tabsize=2,                       
    showstringspaces=false,          
    captionpos=b                     
}


\usepackage[a4paper, margin=3.2cm, top=3.5cm, bottom=4cm]{geometry}

% Modifier - par numéro du TP à rendre
\title{Rapport Intermediaire - Conception Circuit Numérique}
\author{Davy CLARK \& Nolan BUCHET}
\date{Septembre 2025}

\begin{document}

\maketitle

\begin{figure}[H]
    \centering
    \includegraphics[width=1\linewidth]{Graphix/Polytech_Univ.png}
\end{figure}

\vspace{10pt}

\begin{center}
        {\Huge Recepteur de Transmission LIN}
\end{center}

\vspace{10pt}

\begin{figure}[H]
    \centering
    \includegraphics[width=0.4\linewidth]{Image_Base.png}
\end{figure}

\thispagestyle{empty} 

% LOGO IUT
\AddEverypageHook{%
  \begin{textblock*}{100cm}(100cm,100cm) % Ajustez les coordonnées pour décaler l'image
  \begin{tikzpicture}[remember picture, overlay]
    \node[anchor=north east, inner sep=18pt] at (current page.north east) {%
      \includegraphics[scale=0.18]{Graphix/Logo_Universite_de_nantes.png}%
    };
  \end{tikzpicture}%
  \end{textblock*}
}

\AddEverypageHook{%
  \begin{textblock*}{100cm}(100cm,100cm) % Ajustez '15cm' pour déplacer l'image plus à droite
    \begin{tikzpicture}[remember picture, overlay]
      \node[anchor=north west, inner sep=25pt] at (current page.north west) {%
        \includegraphics[scale=0.50]{Graphix/logo_poly.png}%
      };
    \end{tikzpicture}%
  \end{textblock*}
}



% NOM FICHIER
\AddEverypageHook{%
  \begin{textblock*}{1cm}(1cm,1cm)
  \end{textblock*}
  \begin{tikzpicture}[remember picture, overlay]
    \node[anchor=north, inner sep=50pt] at (current page.north) {%    
    {\fontsize{10}{13}\selectfont Compte Rendu Intermediaire Conception Circuit Numérique}
    };
  \end{tikzpicture}%
}

% PAGE
\AddEverypageHook{%
  \begin{textblock*}{1cm}(1cm,1cm)
  \end{textblock*}
  \begin{tikzpicture}[remember picture, overlay]
    \node[anchor=south, inner sep=50pt] at (current page.south) {%
    \thepage/\pageref{LastPage}
    };
  \end{tikzpicture}%
}

% NOM PRENOM
\AddEverypageHook{%
  \begin{textblock*}{1cm}(1cm,1cm)
  \end{textblock*}
  \begin{tikzpicture}[remember picture, overlay]
    \node[anchor=south west, inner sep=50pt] at (current page.south west) {%    
    {\fontsize{10}{13}\selectfont Davy CLARK \& Nolan BUCHET}
    };
  \end{tikzpicture}%
}

% DATE
\AddEverypageHook{%
  \begin{textblock*}{1cm}(1cm,1cm)
  \end{textblock*}
  \begin{tikzpicture}[remember picture, overlay]
    \node[anchor=south east, inner sep=50pt] at (current page.south east) {%    
    {\fontsize{10}{13}\selectfont Septembre 2025}
    };
  \end{tikzpicture}%
}


\pagestyle{empty}
\pagestyle{fancy}

\fancyhead[L]{\large} % Left side of the header (section title)
\fancyhead[R]{\large} % Right side of the header (page number)

% Footer configuration
\fancyfoot[C]{\large} % Centered text in the footer

\renewcommand{\headrulewidth}{1pt} % Header rule thickness
\renewcommand{\footrulewidth}{1pt} % Footer rule thickness
 % Stock pied de page

\tableofcontents

\section{Introduction}

Le projet réalisé dans le cadre de l'enseignement de Conception de Circuits numériques a pour 
objectif de développer des compétences essentielles à la conception de systèmes embarqués, 
notamment la mise au point d'un circuit utilisant un composant logique programmable.
\newline

L'architecture électronique d'un véhicule repose sur une organisation de calculateurs distribués. 
L'exemple retenu s'inspire du fonctionnement d'un calculateur embarqué dans la portière d'une 
automobile, chargé de la gestion des rétroviseurs et des vitres électriques.
\newline

Dans ce contexte, deux sous-ensembles sont distingués : 
\begin{itemize}
    \item un sous-ensemble de supervision, qui génère les commandes pour les moteurs des rétroviseurs et des vitres électriques,
    \item un sous-ensemble d'interface, assurant la communication entre le sous-ensemble de supervision et les autres calculateurs du véhicule.
\end{itemize}

Ce rapport se concentre exclusivement sur ce second sous-ensemble, l'interface microprocesseur, afin d'étudier son rôle et sa conception.

L'un des objectifs principaux est d'appréhender la conception du circuit via la méthode MCSE (Méthode de Conception de Systèmes Électroniques), en mettant l'accent sur les étapes de spécifications et de conception. Le déroulement du rapport suit la logique du diagramme en Y.

\begin{figure}[H]
    \centering
    \includegraphics[width=0.8\linewidth]{images/Intro/Architecture_LIN.png}
    \caption{Exemple d'architecture d'un réseau dans un véhicule}
    \label{fig:placeholder}
\end{figure}

\section{Protocole LIN}

\subsubsection*{Architecture}  
Le bus LIN est un système \textit{mono-maître et multi-esclaves}. Un seul maître initie toutes les communications, ce qui rend inutile toute fonction d'arbitrage. Le nombre d'esclaves n'est pas limité par la norme mais dépend des contraintes électriques. L'architecture est dite \textit{flexible}, car on peut ajouter des nœuds esclaves sans modifier les nœuds existants.  

\begin{figure}[H]
    \centering
    \includegraphics[width=0.8\linewidth]{images//Intro/Architecture_LIN_ME.png}
    \caption{Exemple d'architecture LIN}
    \label{fig:placeholder}
\end{figure}

\subsubsection*{Connexion}  
Le bus est constitué d'une \textit{seule ligne} reliée à chaque nœud par une sortie à collecteur ouvert. Le maître utilise une résistance de tirage de \SI{1}{k\ohm}, tandis que chaque esclave utilise \SI{30}{k\ohm}. La ligne est au niveau \textit{récessif (1)} lorsqu'aucun nœud ne force l'état, et au niveau \textit{dominant (0)} dès qu'au moins un nœud impose ce niveau.  

\begin{figure}[H]
    \centering
    \includegraphics[width=0.5\linewidth]{images//Intro/Connection_LIN.png}
    \caption{Connexion physique d'un noeud à la ligne LIN}
    \label{fig:placeholder}
\end{figure}

\subsubsection*{Vitesse de transmission}  
Le débit varie de \SI{1}{kbit/s} à \SI{20}{kbit/s}, fixé pour une architecture donnée. Trois vitesses sont recommandées :  
\begin{itemize}
    \item \textbf{Lente} : 2400 bit/s,  
    \item \textbf{Moyenne} : 9600 bit/s,  
    \item \textbf{Rapide} : 19200 bit/s.  
\end{itemize}

\subsubsection*{Communications et trames}  
Les messages LIN sont composés de plusieurs champs :  
\begin{itemize}
    \item \textit{Synchronisation Break} : marque le début du message,  
    \item \textit{Synchronisation Field} : alignement des horloges (valeur 0x55),  
    \item \textit{Identification Field} : contenu et longueur des données, avec contrôle de parité,  
    \item \textit{Data Field} : octets d'information transmis du LSB vers le MSB,  
    \item \textit{Checksum Field} : somme de contrôle des données (modulo 256 inversée).  
\end{itemize}

\begin{figure}[H]
    \centering
    \includegraphics[width=0.8\linewidth]{images//Intro/Trame_LIN.png}
    \caption{Type de Trame Protocol LIN}
    \label{fig:placeholder}
\end{figure}

Une communication peut être de deux types :  
\begin{itemize}
    \item \textbf{Écriture} : le maître envoie l'intégralité du message,  
    \item \textbf{Lecture} : le maître envoie seulement l'entête, puis reçoit la réponse de l'esclave.  
\end{itemize}




\section{Cahier des Charges}

Le projet se concentre sur une partie restreinte du récepteur LIN, uniquement pour la réception de trames de type \textbf{« écriture »}, avec une \textbf{entrée LIN unique} et une vitesse fixée à \SI{19200}{bit/s}. La distinction maître/esclave et la connexion physique complète ne sont pas traitées.

\subsubsection*{Limitations et simplifications :}
\begin{itemize}
    \item Pas de gestion de perte d'octets,
    \item Vérification des bits start/stop par un seul échantillon,
    \item Pas de contrôle de parité ni de vérification du checksum.
\end{itemize}

\subsubsection*{Fonctionnalités attendues :}
\begin{itemize}
    \item Conversion série → parallèle (données de 8 bits) pour un microprocesseur,
    \item Possibilité de \textbf{filtrer les messages} grâce à un registre de comparaison \texttt{SelAdr} (8 bits),
    \item Signalisation de fin de réception (\texttt{M\_Received}) uniquement si l'identifiant reçu correspond à \texttt{SelAdr},
    \item Réinitialisation des compteurs et effacement des messages non valides.
\end{itemize}

\subsubsection*{Gestion des messages :}
\begin{itemize}
    \item Un seul message peut être stocké à la fois (FIFO),
    \item Les octets doivent être accessibles dans leur ordre d'arrivée, même si le message est encore en cours de réception,
    \item Tous les octets doivent être mémorisés, indépendamment du filtrage,
    \item Le récepteur doit déterminer la fin du message et l'indiquer au microprocesseur.
\end{itemize}

\subsubsection*{État du récepteur :}  
Accessible par registre ( ETAT ) à tout moment, il doit indiquer :
\begin{itemize}
    \item si un message a été reçu (après filtrage),
    \item le nombre d'octets reçus,
    \item les erreurs simples de réception (bits START/STOP, durée du \textit{synchro break}).
\end{itemize}

Après lecture du registre d'état, les champs sont réinitialisés (sauf le compteur d'octets reçus).

\subsubsection*{Contraintes supplémentaires :}
\begin{itemize}
    \item Interface physique avec le microprocesseur imposée,
    \item Caractéristiques fonctionnelles, physiques et temporelles définies,
    \item Temps d'échanges précisés pour assurer la compatibilité avec l'environnement.
\end{itemize}

\begin{figure}[H]
    \centering
    \includegraphics[width=0.8\linewidth]{images/CDC/Compo_concevoir.png}
    \caption{Interface microprocesseur associée au circuit à concevoir}
    \label{fig:placeholder}
\end{figure}

\begin{figure}[H]
    \centering
    \includegraphics[width=0.8\linewidth]{images/CDC/Chrono.png}
    \caption{Chronogrammes des échanges entre le circuit et son environnement}
    \label{fig:placeholder}
\end{figure}

% Page 2 TD DAVY
\begin{figure}[H]
    \centering
    \includegraphics[width=0.8\linewidth]{images/CDC/Schema_Fifi_etat.pdf}
    \caption{Schema Conception Registre interne Système}
    \label{fig:placeholder}
\end{figure}



\section{Description des différentes spécifications définies en travaux dirigés}

\subsection*{Objectif}

Les spécifications ont pour objectif de définir le comportement attendu du système, autrement dit, 
ce qui doit être réalisé en réponse au cahier des charges. Elles constituent l'expression formelle 
des besoins fonctionnels, en se plaçant du point de vue de l'environnement du système,
c'est-à-dire, tout ce qui est externe au circuit mais interagit avec lui.
\newline

La phase de spécification représente la première étape de la conception d'un circuit. Elle adopte
une approche boîte noire : on s'intéresse uniquement à ce que le système doit faire, sans se 
préoccuper des solutions techniques internes ni du comment il fonctionnera. Les spécifications 
sont donc indépendantes de la technologie utilisée.
\newline

Au départ, nous avions pour indication de créer un bloc capable de communiquer à la fois avec 
le système de trame LIN et avec le microcontrôleur. Afin de simplifier notre étude et nos 
spécifications, nous avons choisi de diviser le système en deux parties : la réception de 
trame, qui se chargera de gérer la réception des données octet par octet, et un autre bloc dédié à 
la communication avec le microcontrôleur.
\newline

% Feuille 1 NOLAN 
\begin{figure}[H]
   \centering
   \includegraphics[width=0.8\linewidth]{images/CDC/Schema_cdc_final.pdf}
   \caption{Interface microprocesseur associée au circuit à concevoir}
   \label{fig:placeholder}
\end{figure}
    

\subsection{Interface Microprocesseur}

D'après les données du cahier des charges, nous pouvons définir tous les signaux nécessaires pour 
le système réception trame : 

\begin{center}
\renewcommand{\arraystretch}{1.2} % espace vertical
\small % pour uniformiser la taille du texte
\begin{tabularx}{\textwidth}{|c||c|c|X|}
    \hline			
    \textbf{Signaux} & \textbf{Mode} & \textbf{Type} & \textbf{Description}  \\ \hline 
    D07 & INOUT & \texttt{STD\_LOGIC\_VECTOR(7 DOWNTO 0)} & Bus de données \\
    nCS & IN & \texttt{STD\_LOGIC} & Chip Select \\
    RnW & IN & \texttt{STD\_LOGIC} & Operation Lecture / Ecriture \\
    CnD & IN & \texttt{STD\_LOGIC} & Operation Contrôle / Données \\
    nCLR & IN & \texttt{STD\_LOGIC} & Réinitialisation \\
    M\_Received & OUT & \texttt{STD\_LOGIC} & Fin de réception de la trame \\
    H & IN & \texttt{STD\_LOGIC} & Horloge \\
    \hline  
\end{tabularx}
\end{center}


\subsection{Reception Trame}

D'après les données du cahier des charges, nous pouvons définir tous les signaux nécessaires pour le système interface Reception Trame : 

\begin{center}
\renewcommand{\arraystretch}{1.2} % pour un peu plus d'espace vertical
\small % pour uniformiser la taille du texte
\begin{tabularx}{\textwidth}{|c||c|c|X|}
    \hline			
    \textbf{Signaux} & \textbf{Mode} & \textbf{Type} & \textbf{Description}  \\ \hline 
    LIN & IN & \texttt{STD\_LOGIC} & Reception de la trame LIN \\
    \hline  
\end{tabularx}
\end{center}

\subsection{FIFO}

Pour le moment, nous savons qu'il s'agit d'un registre de stockage fonctionnant en mode FIFO, destiné à mémoriser les données de réception d'une trame LIN : 

\begin{center}
\renewcommand{\arraystretch}{1.2} % pour un peu plus d'espace vertical
\small % pour uniformiser la taille du texte
\begin{tabularx}{\textwidth}{|c||c|c|X|}
    \hline
    \textbf{Signaux} & \textbf{Mode} & \textbf{Type} & \textbf{Description} \\ \hline
    OctetRecu & IN & \texttt{STD\_LOGIC\_VECTOR(7 DOWNTO 0)} & Signal Entrée \\
    OctetLu & OUT & \texttt{STD\_LOGIC\_VECTOR(7 DOWNTO 0)} & Signal Sortie \\
    \hline
\end{tabularx}
\end{center}

\subsection{ETAT}

Ce registre peut être considéré comme le registre de log de la trame. Grâce au cahier des charges, nous connaissons précisément ses fonctionnalités :

\begin{center}
\renewcommand{\arraystretch}{1.2} % espace vertical
\small % pour uniformiser la taille du texte
\begin{tabularx}{\textwidth}{|c||c|c|X|}
    \hline			
    \textbf{Signaux} & \textbf{Mode} & \textbf{Type} & \textbf{Description}  \\ \hline 
    Erreur\_Start & IN & \texttt{STD\_LOGIC} & Bit d'erreur de Start \\
    Erreur\_Stop & IN & \texttt{STD\_LOGIC} & Bit d'erreur de Stop \\
    Erreur\_SynchroBreak & IN & \texttt{STD\_LOGIC} & Bit d'erreur de Synchro Break \\
    NbOctetReceived & IN & \texttt{STD\_LOGIC\_VECTOR(3 DOWNTO 0)} & Nombre d'octets reçus \\
    MessageReceived\_SET & IN & \texttt{STD\_LOGIC} & Indicateur de trame reçue \\
    EtatLu & OUT & \texttt{STD\_LOGIC\_VECTOR(7 DOWNTO 0)} & Octet d'information de la trame \\
    \hline  
\end{tabularx}
\end{center}


\section{Description et justification de la structure fonctionnelle}

\subsubsection*{Objectifs}

Cette section présente l'organisation du système en blocs fonctionnels et décrit leurs 
interactions. Chaque bloc (réception de trame, FIFO, registre d'état, interface microprocesseur) 
est expliqué dans son rôle et sa contribution au fonctionnement global. L'objectif est de montrer 
comment les fonctionnalités spécifiées sont réparties de manière logique pour répondre au cahier 
des charges.
\newline

Dans cette section, nous établissons une structure fonctionnelle indépendante de la technologie.

\begin{figure}[H]
    \centering
    \includegraphics[width=0.8\linewidth]{images/inter/Structure_Fonc_Circuit.pdf}
    \caption{Structure fonctionnelle du système}
    \label{fig:placeholder}
\end{figure}

Pour le moment, nous nous sommes limités à deux blocs principaux : l'interface microprocesseur et 
la réception des trames. Ces deux blocs sont connectés à un bloc d'échange, qui permet la 
communication entre eux. Ce bloc assure l'interprétation des échanges avec le microprocesseur ainsi 
que la gestion des deux registres de données.
\newline

% page 12
\begin{figure}[H]
    \centering
    \includegraphics[width=0.8\linewidth]{images/inter/Schema_base_circuit.pdf}
    \caption{Architecture du système de réception de trame LIN V1}
    \label{fig:placeholder}
\end{figure}

Nous pouvons également représenter l'automate du système de communication avec le processeur : 

\begin{figure}[H]
    \centering
    \includegraphics[width=0.8\linewidth]{images/inter/Echange_Processeur.pdf}
    \caption{Échanges avec le processeur}
    \label{fig:placeholder}
\end{figure}


Dans notre démarche d'optimisation du système, nous nous sommes rendu compte que le bloc d'échange 
microprocesseur pouvait être intégré au bloc d'interface. Cela permettra de supprimer des signaux 
supplémentaires, inutiles et encombrants.
\newline

%page 13
\begin{figure}[H]
    \centering
    \includegraphics[width=0.8\linewidth]{images/inter/Schema_avance_circuit.pdf}
    \caption{Architecture du système de réception de trame LIN V2}
    \label{fig:placeholder}
\end{figure}

Par la suite, nous avons décidé d'ajouter deux blocs de registres internes : le registre de stockage 
des données de la trame LIN (FIFO) et le registre d'état interne (ETAT), qui permet de connaître les 
informations sur l'état de la réception.
\newline

%page 16
\begin{figure}[H]
    \centering
    \includegraphics[width=0.8\linewidth]{images/inter/Schema_Final.pdf}
    \caption{Description finale du circuit}
    \label{fig:placeholder}
\end{figure}

Comme indiqué ci-dessus, dans le schéma de notre système global, nous avons décidé d'ajouter 
certains signaux internes entre les blocs de registres et les interfaces.
\newline

\subsubsection*{FIFO}

\begin{center}
\renewcommand{\arraystretch}{1.2} % espace vertical
\small % pour uniformiser la taille du texte
    \begin{tabularx}{\textwidth}{|c||c|c|X|}
     \hline			
       \textbf{Signaux} & \textbf{Mode} & \textbf{Type} & \textbf{Description}  \\ \hline 
       OctetRecu\_WR & IN & \texttt{STD\_LOGIC} & Read / Write opération \\
       OctetRecu\_RST & IN & \texttt{STD\_LOGIC} & Réinitialisation des données reçue \\
       OctetLu\_RD & IN & \texttt{STD\_LOGIC} & Sélection de la mémoire (Control / Data) \\
     \hline  
    \end{tabularx}
\end{center}

La définition de ces nouveaux signaux permet de gérer de manière précise le fonctionnement de la 
FIFO. OctetRecu\_WR est un signal d'écriture qui déclenche l'enregistrement d'un octet reçu dans 
la mémoire FIFO, garantissant que chaque donnée entrante est correctement stockée. OctetRecu\_RST 
est un signal de réinitialisation qui vide complètement la FIFO et remet à zéro tous les compteurs 
internes, permettant ainsi de reprendre la réception de données sans risque de corruption ou de 
chevauchement. OctetLu\_RD est un signal de lecture qui active l'accès aux données stockées dans 
la FIFO et permet leur transfert vers le microprocesseur ou d'autres blocs du système. Ensemble, 
ces signaux assurent une communication fiable, synchronisée et organisée entre l'interface 
microprocesseur et le registre de réception, tout en facilitant la gestion des flux de données 
entrants.
\newline

\subsubsection*{ETAT}

\begin{center}
\renewcommand{\arraystretch}{1.2} % espace vertical
\small % pour uniformiser la taille du texte
    \begin{tabularx}{\textwidth}{|c||c|c|X|}
     \hline			
       \textbf{Signaux} & \textbf{Mode} & \textbf{Type} & \textbf{Description}  \\ \hline 
       NbOctetRecu\_RST & IN & \texttt{STD\_LOGIC} & Réinitialisation du compteur d'octets \\
       DecNbOctet & IN & \texttt{STD\_LOGIC} & Flag de lecture pour FIFO \\
       EtatLu\_RST & IN & \texttt{STD\_LOGIC} & Reset de l'état lu \\
     \hline  
    \end{tabularx}
\end{center}

La définition de ces signaux du bloc ETAT permet de contrôler et de suivre l'état interne de la 
réception des données. NbOctetRecu\_RST réinitialise le compteur d'octets reçus, assurant un 
suivi précis du nombre de données traitées. DecNbOctet agit comme un indicateur de lecture pour 
la FIFO, signalant quand un octet peut être lu et transféré, ce qui permet une gestion correcte 
des flux de données. EtatLu\_RST permet de réinitialiser les informations d'état lues, garantissant 
que le système dispose toujours d'une représentation exacte de l'état actuel de la réception. 
Ensemble, ces signaux assurent une supervision fiable et synchronisée des opérations internes, 
facilitant le contrôle et la gestion du flux de données dans le système.


\section{Description et justification de la solution architectureale obtenue pour le circuit}

\subsubsection*{Objectifs}

Une fois les algorithmes fonctionnels définis, la description fonctionnelle interne du circuit 
est considérée comme complète. Cette description reste indépendante de la technologie utilisée 
et ne prend pas nécessairement en compte les réalités physiques, telles que les interférences 
entre signaux et entités.
\newline

La phase de conception architecturale consiste alors à intégrer les interfaces physiques, à analyser 
les opportunités de simplification des algorithmes, et à définir la stratégie d'implantation du 
circuit.
\newline

Dans cette partie, nous nous intéressons à l'architecture matérielle à partir de la solution 
fonctionnelle et à la description du circuit au niveau RT (Register Transfer).  
\newline

\begin{itemize}
    \item Introduction des interfaces physiques
    \item Identification des ressources logiques de stockage
    \item Description structurelle du circuit au niveau RT
    \item Écriture du comportement du circuit au niveau RT
\end{itemize}

\vspace{20px}

\subsection{Horloge}

Nous commençons par l'une des parties les plus simples du système : la gestion de l'horloge.

La vitesse de l'horloge est déterminée à partir du pas d'échantillonnage \(N\), défini par :

\[
N = \frac{T_{bit}}{T_{processeur}}
\]

Dans notre cas, le cycle de lecture/écriture spécifié dans le cahier des charges est en moyenne de 100~ns, avec une vitesse de transmission de 19\,200~bit/s. Cela donne :

\[
N = \frac{52~\mu s}{100~ns} = 520
\]

Pour notre implémentation, nous choisissons \(N = 2048\), une valeur arbitraire qui facilite la synchronisation et le traitement interne.

\subsection{Architecture Reception Trame}

Pour rappel voici tous les signaux associés à ce bloc : 
\newline

\begin{center}
\renewcommand{\arraystretch}{1.2} % espace vertical
\small % pour uniformiser la taille du texte
    \begin{tabularx}{\textwidth}{|c||c|c|X|}
     \hline			
       \textbf{Signaux} & \textbf{Mode} & \textbf{Type} & \textbf{Description}  \\ \hline 
       LIN & IN & \texttt{STD\_LOGIC} & Bus de données d'entrée \\
       SelAdr & IN & \texttt{STD\_LOGIC\_VECTOR(7 DOWNTO 0)} & Sélection Adrress Composant \\
       OctetRecu & OUT & \texttt{STD\_LOGIC\_VECTOR(7 DOWNTO 0)} & Bus de données de sortie \\
       OctetRecu\_WR & OUT & \texttt{STD\_LOGIC} & Read / Write opération \\
       OctetRecu\_RST & OUT & \texttt{STD\_LOGIC} & Réinitialisation des données reçues \\
       Erreur\_Start & OUT & \texttt{STD\_LOGIC} & Bit d'erreur de Start \\
       Erreur\_Stop & OUT & \texttt{STD\_LOGIC} & Bit d'erreur de Stop\\
       Erreur\_SynchroBreak & OUT & \texttt{STD\_LOGIC} & Bit d'erreur de Synchro Break\\
       IncNbOctet & OUT & \texttt{STD\_LOGIC} & Flag de reception pour lecture \\
       MessageReceived\_SET & OUT & \texttt{STD\_LOGIC} & Indicateur de trame reçue \\
       NbOctetRecu\_RST & OUT & \texttt{STD\_LOGIC} & Réinitialisation du compteur d'octets \\
     \hline  
    \end{tabularx}
\end{center}

Pour le bloc de réception de trames, nous implémentons une machine séquentielle qui permet de distinguer une partie opérative et une unité de commande, assurant ainsi la gestion efficace de la réception de ce type de trame.

%Page 23 shema bas 
\begin{figure}[H]
    \centering
    \includegraphics[width=0.8\linewidth]{images/inter/Machine_Seq_Reception_trame.pdf}
    \caption{Machine Sequentielle Reception Trame}
    \label{fig:placeholder}
\end{figure}

Nous ajoutons des variable interne pour la communication entre les deux bloc de la machine afin d'obtenir un syteme correcte : 

\begin{table}[h!]
\centering
\resizebox{\textwidth}{!}{%
\begin{tabular}{|c|c|c|c|c|}
\hline
\textbf{Variable} & \textbf{Taille (bit)} & \textbf{Opération} & \textbf{Opérateur} & \textbf{Signaux de contrôle} \\ \hline
n & $\log_2(N)$ & décrémentation, initialisation à $N-1$ ou $N/2$ & décompteur, Mux & n\_Load, n\_En, n\_select \\ \hline
NbTbit & 4 & décrémentation, initialisation à 13 ou 8 & décompteur, Mux & NBTbit\_Load, NBTbit\_en, NBTbit\_select \\ \hline
Identifier & 8 & sauvegarde & registre 8 bits & Identifier\_en \\ \hline
OctetsReçus & 8 & décalage bit à bit & registre à décalage & OctetReçu\_en \\ \hline
NbDataField & 3 & décrémentation, initialisation à 1, 3 ou 7 & décompteur, décodeur & NBdatafield\_en, NBdatafield\_load \\ \hline
\end{tabular}%
}
\end{table}


% Page 21 shema haut 
\begin{figure}[H]
    \centering
    \includegraphics[width=0.95\linewidth]{images/inter/Structure_Reception_trame.pdf}
    \caption{Structure partie Opérative Reception Trame}
    \label{fig:placeholder}
\end{figure}

Dans cette partie nous retrouvons les différents compteurs et registres nécessaires au fonctionnement 
de la réception de trame. Pour prendre un exemple le premier en haut à gauche représente le 
décompteur n qui permet de décompter jusqu'à N-1 ou N/2 pour l'échantillonnage. 
Le choix de N-1 ou N/2 dépend de l'utilisation, soit N/2 pour la detection de l'état au bout de 
un demi Tbit ou N-1 pour la detection de l'état à la fin de un Tbit ( Potentiel Front ).

% \newline

Pour la partie commande, nous avons conçu et implémenté un automate afin de gérer les différentes 
séquences de contrôle : 

% \newline

%page 21 & 22 TD DAVY 
\begin{figure}[H]
    \centering
    \includegraphics[width=0.6\linewidth]{images/inter/Automate_Reception_trame.pdf}
    \caption{Structure partie Commande Reception Trame}
    \label{fig:placeholder}
\end{figure}

\subsection*{Description de l'automate}
Cet automate décrit un processus de réception de trame LIN (Local Interconnect Network), organisé en états et transitions conditionnées par des signaux de synchronisation, des compteurs et des champs de données. Il commence par une phase d'\textbf{attente}, suivie d'une \textbf{synchronisation sur break} où la ligne (LinSynchro) est surveillée pour détecter le début de la trame. Des erreurs de synchro ou de bit de start peuvent être signalées via des flags comme \texttt{ErreurBitSynchroBreak\_set} ou \texttt{ErreurBitStart\_set}.

Ensuite, l'automate passe à la réception des différents champs de la trame :
\begin{itemize}
    \item \textbf{Synchro} (octet de synchronisation),
    \item \textbf{Identifier} (champ d'identifiant),
    \item \textbf{Data field} (données, avec gestion du compteur \texttt{NbDataField}),
    \item \textbf{Checksum} (vérification de l'intégrité).
\end{itemize}

À chaque octet, le système contrôle le bit de start, les données, et le bit de stop, en décrémentant 
les compteurs comme \texttt{NbTbit} ou \texttt{NbOctetsRecus}. Selon que l'identifiant reçu 
correspond ou non à l'adresse sélectionnée (\texttt{SelAdr}), l'automate valide la trame 
(\texttt{M\_Received\_set}) ou la rejette. Des réinitialisations (\texttt{rst}) ont lieu après 
réception complète ou en cas d'erreur.

Ce processus assure une réception séquentielle et robuste, typique des protocoles série, avec gestion 
d'erreurs et validation conditionnelle des trames.

Cet automate peut être représenté sous forme de machine de Mealy, ce qui simplifie l'écriture du 
code VHDL : 

%page 25 TD DAVY
\begin{figure}[H]
    \centering
    \includegraphics[width=0.8\linewidth]{images/inter/MEALY_Reception_trame.pdf}
    \caption{Machine de MEALY Unité de Commande Reception Trame}
    \label{fig:placeholder}
\end{figure}

\subsection{Architecture Interface MicroProcesseur}

Pour rappel voici tous les signaux associés à ce bloc : 
\newline

\begin{center}
\renewcommand{\arraystretch}{1.2} % espace vertical
\small % pour uniformiser la taille du texte
    \begin{tabularx}{\textwidth}{|c||c|c|X|}
     \hline		
       \textbf{Signaux} & \textbf{Mode} & \textbf{Type} & \textbf{Description}  \\ \hline 
       D07 & INOUT & \texttt{STD\_LOGIC\_VECTOR(7 DOWNTO 0)} & Bus de données \\
       nCS & IN & \texttt{STD\_LOGIC} & Chip Select \\
       RnW & IN & \texttt{STD\_LOGIC} & Opération Lecture / Écriture \\
       CnD & IN & \texttt{STD\_LOGIC} & Opération Contrôle / Données \\
       nCLR & IN & \texttt{STD\_LOGIC} & Réinitialisation \\
       M\_Received & OUT & \texttt{STD\_LOGIC} & Fin de réception de la trame \\
       H & IN & \texttt{STD\_LOGIC} & Horloge \\
       EtatLu & IN & \texttt{STD\_LOGIC\_VECTOR(3 DOWNTO 0)} & Octet d'information de la Trame \\
       DecNbOctet & OUT & \texttt{STD\_LOGIC} & Flag de lecture pour FIFO \\
       EtatLu\_RST & OUT & \texttt{STD\_LOGIC} & Reset de l'état lu \\
       OctetLu & IN & \texttt{STD\_LOGIC\_VECTOR(7 DOWNTO 0)} & Bus de données de sortie \\
       OctetLu\_RD & OUT & \texttt{STD\_LOGIC} & Sélection de la mémoire (Control / Data) \\
     \hline  
    \end{tabularx}
\end{center}

Pour le bloc d'interface Microprocesseur, nous implémentons une machine séquentielle qui permet 
de distinguer une partie opérative et une partie commande, assurant ainsi la gestion efficace du 
controle du systeme et la lecture des différentes informations.

%page 24 BAS TD DAVY schema bas
\begin{figure}[H]
    \centering
    \includegraphics[width=0.8\linewidth]{images/inter/Machine_Seq_Interface_Micro.pdf}
    \caption{Machine Sequentielle Reception Trame}
    \label{fig:placeholder}
\end{figure}

% Page 24 shema haut 
\begin{figure}[H]
    \centering
    \includegraphics[width=0.8\linewidth]{images/inter/Structure_Interface_Micro.pdf}
    \caption{Structure partie Opérative Interface Microprocesseur}
    \label{fig:placeholder}
\end{figure}

Pour la partie opérative de l'interface microprocesseur, nous obtenons un schéma simple de 
multiplexeur permettant de choisir quelles données mettre dans le vecteur D07 en sortie, 
soit celles provenant de la FIFO, soit celles de l'État interne.
\newline

Pour la partie commande, nous avons conçu et implémenté un automate afin de gérer les différentes 
séquences de contrôle : 
\newline


\begin{figure}[H]
    \centering
    \includegraphics[width=0.8\linewidth]{images/inter/Automate_Interface_Micro.pdf}
    \caption{Structure partie Commande Interface Microprocesseur}
    \label{fig:placeholder}
\end{figure}

Cet automate peut être représenté sous forme de machine de Mealy, ce qui simplifie l'écriture du code VHDL : 

\begin{figure}[H]
    \centering
    \includegraphics[width=0.8\linewidth]{images/inter/MEALY_Interface_Micro.pdf}
    \caption{Machine de MEALY Unité de Commande Interface Micro}
    \label{fig:placeholder}
\end{figure}

\subsection{Architecture FIFO}

Pour rappel voici tous les signaux associés à ce bloc : 
\newline

\begin{center}
\renewcommand{\arraystretch}{1.2} % espace vertical
\small % pour uniformiser la taille du texte
    \begin{tabularx}{\textwidth}{|c||c|c|X|}
     \hline				
       \textbf{Signaux} & \textbf{Mode} & \textbf{Type} & \textbf{Description}  \\ \hline 
       OctetRecu & IN & \texttt{STD\_LOGIC\_VECTOR(7 DOWNTO 0)} & Bus de données d'entrée \\
       OctetRecu\_WR & IN & \texttt{STD\_LOGIC} & Read / Write opération \\
       OctetRecu\_RST & IN & \texttt{STD\_LOGIC} & Réinitialisation des données reçues \\
       OctetLu & OUT & \texttt{STD\_LOGIC\_VECTOR(7 DOWNTO 0)} & Bus de données de sortie \\
       OctetLu\_RD & IN & \texttt{STD\_LOGIC} & Sélection de la mémoire (Control / Data) \\
     \hline  
    \end{tabularx}
\end{center}

Étant donné que ce système est moins complexe que les deux interfaces précédentes, nous avons choisi de ne pas le réaliser sous forme de machine séquentielle et de le représenter directement comme un bloc, comme montré ci-dessous : 
\newline

%page 14 haut
\begin{figure}[H]
    \centering
    \includegraphics[width=0.8\linewidth]{images/inter/Implementation_FIFO.pdf}
    \caption{Implémentation de la FIFO}
    \label{fig:placeholder}
\end{figure}

\subsection{Implémentation de l'État Interne}

Pour rappel voici tous les signaux associés à ce bloc : 
\newline

\begin{center}
\renewcommand{\arraystretch}{1.2} % espace vertical
\small % pour uniformiser la taille du texte
    \begin{tabularx}{\textwidth}{|c||c|c|X|}
     \hline				
       \textbf{Signaux} & \textbf{Mode} & \textbf{Type} & \textbf{Description}  \\ \hline 
       Erreur\_Start & IN & \texttt{STD\_LOGIC} & Bit d'erreur de Start \\
       Erreur\_Stop & IN & \texttt{STD\_LOGIC} & Bit d'erreur de Stop\\
       Erreur\_SynchroBreak & IN & \texttt{STD\_LOGIC} & Bit d'erreur de Synchro Break\\
       IncNbOctet & IN & \texttt{STD\_LOGIC} & Flag de reception pour lecture \\
       MessageReceived\_SET & IN & \texttt{STD\_LOGIC} & Indicateur de trame reçue \\
       NbOctetRecu\_RST & IN & \texttt{STD\_LOGIC} & Réinitialisation du compteur d'octets \\
       EtatLu & OUT & \texttt{STD\_LOGIC\_VECTOR(7 DOWNTO 0)} & Octet d'information de la Trame \\
       DecNbOctet & IN & \texttt{STD\_LOGIC} & Flag de lecture pour FIFO \\
       EtatLu\_RST & IN & \texttt{STD\_LOGIC} & Reset de l'état lu \\
     \hline  
    \end{tabularx}
\end{center}

Étant donné que ce système présente une complexité similaire et une facilité d'implémentation comparable, nous le conservons également sous forme de bloc, comme montré ci-dessous : 
\newline

%page 14 bas
\begin{figure}[htbp]
    \centering
    \begin{subfigure}[b]{0.49\textwidth}
        \centering
        \includegraphics[width=\textwidth]{images/inter/Implementation_ETAT_Erreur.pdf}
        \caption{Mémorisation de l'erreur}
        \label{fig:placeholder}
    \end{subfigure}
    \hfill
    \begin{subfigure}[b]{0.49\textwidth}
        \centering
        \includegraphics[width=\textwidth]{images/inter/Implementation_ETAT_NbOctets.pdf}
        \caption{Nombre d'octets reçus}
        \label{fig:placeholder}
    \end{subfigure}
    \caption{implémentation de l'état interne en représentation structurelle au niveau RT}
    \label{fig:placeholder} % This is the label for the global figure
\end{figure}


\section{Présentation du fonctionnement des fonctions}

\subsection{Interface MicroProcesseur}

Dans cette partie, nous avons initié une séance de travaux pratiques pour nous familiariser avec 
le logiciel HDL Designer. Le programme «Interface Microprocesseur», préalablement implémenté par 
les enseignants, respecte strictement les données présentées dans le TD et développées dans les 
sections précédentes du rapport. Nous allons l’étudier en détail afin de démontrer sa correspondance 
avec le modèle théorique.
\newline

Pour rappel, l’interface Microprocesseur a été conçue selon une machine séquentielle, tandis que 
la partie commande a été développée sur le modèle d’une machine de Mealy. Le code présenté respecte 
rigoureusement la structure des blocs : réseau combinatoire d’entrée, réseau combinatoire de sortie 
et registres correspondant à la machine à états.
\newline

\subsubsection{Réseau Combinatoire d’Entrée}

\begin{lstlisting}[style=VHDLStyle, caption={Reseau Cominatoire d'entrée}]
InputProc_Synchro :  PROCESS(H, nRST)
BEGIN
  IF (nRST='0') THEN 
    nCS_Synchro <= '1';
    RnW_Synchro <= '1';
    CnD_Synchro <= '1';
    D07_Synchro <= (others => '0');
  ELSIF (H'EVENT AND H='1') THEN
    nCS_Synchro <= nCS;
    RnW_Synchro <= RnW;
    CnD_Synchro <= CnD;
    D07_Synchro <= D07;
  END IF;
END PROCESS InputProc_Synchro;
\end{lstlisting}

Ce bloc VHDL gère la synchronisation des signaux provenant du microprocesseur. Le processus \texttt{InputProc\_Synchro} lit les signaux d’entrée à chaque front montant de l’horloge \texttt{H} et les initialise lors de la mise à zéro \texttt{nRST}. Les signaux synchronisés (\texttt{nCS\_Synchro, RnW\_Synchro, CnD\_Synchro, D07\_Synchro}) sont ensuite utilisés par le reste de l’interface.

\subsubsection{Réseau Combinatoire de Sortie}

\begin{lstlisting}[style=VHDLStyle, caption={Reseau Cominatoire de Sortie}]
OutputProc_Comb : PROCESS(nCS_Synchro, CnD_Synchro, RnW_Synchro, EtatCourant, OctetLu, EtatLu)
BEGIN
  D07 <= (others => 'Z');
  OctetLu_RD <= '0';
  EtatLu_RST <= '0';
  DecNbOctet <= '0';
  CASE EtatCourant IS
    WHEN Attente =>
      IF (nCS_Synchro='0' AND CnD_Synchro='0' AND RnW_Synchro='1') THEN
        OctetLu_RD <= '1';
      END IF;
    WHEN LectureData =>
      D07 <= OctetLu;
      IF (nCS_Synchro='1') THEN
        DecNbOctet <= '1';
      END IF;
    WHEN LectureEtat =>
      D07 <= EtatLu;
      IF (nCS_Synchro='1') THEN
        EtatLu_RST <= '1';
      END IF;
    WHEN EcritureFiltre =>   
    END CASE;
END PROCESS OutputProc_Comb;
\end{lstlisting}

Le processus \texttt{OutputProc\_Comb} contrôle la sortie des données et des états vers le microprocesseur. Il met à jour les signaux \texttt{D07, OctetLu\_RD, EtatLu\_RST, DecNbOctet} en fonction de l’état courant de la machine et des signaux synchronisés d’entrée. La logique combinatoire assure la correspondance entre les actions de lecture/écriture et l’état de la machine.

\subsubsection{Registres et Machine à États}

\begin{lstlisting}[style=VHDLStyle, caption={Registres}]
ClockedProc : PROCESS(H, nRST)
BEGIN
  IF (nRST='0') THEN
    EtatCourant <= Attente;
  ELSIF (H'EVENT AND H='1') THEN
    EtatCourant <= EtatSuivant;
  END IF;
END PROCESS ClockedProc;

NextStateProc : PROCESS(nCS_Synchro, CnD_Synchro, RnW_Synchro, EtatCourant)
BEGIN
  EtatSuivant <= EtatCourant;
  CASE EtatCourant IS
  WHEN Attente =>
    IF (nCS_Synchro='0' AND CnD_Synchro='0' AND RnW_Synchro='1') THEN
      EtatSuivant <= LectureData;
    ELSIF (nCS_Synchro='0' AND CnD_Synchro='1' AND RnW_Synchro='1') THEN
      EtatSuivant <= LectureEtat;
    ELSIF (nCS_Synchro='0' AND CnD_Synchro='0' AND RnW_Synchro='0') THEN
      EtatSuivant <= EcritureFiltre;
    ELSE
      EtatSuivant <= Attente;
    END IF;
    WHEN LectureData =>
      IF (nCS_Synchro='1') THEN
        EtatSuivant <= Attente;
      ELSE
        EtatSuivant <= LectureData;
      END IF;
    WHEN LectureEtat =>
      IF (nCS_Synchro='1') THEN
        EtatSuivant <= Attente;
      ELSE
        EtatSuivant <= LectureEtat;
      END IF;
    WHEN EcritureFiltre =>
      IF (nCS_Synchro='1') THEN
        EtatSuivant <= Attente;
      ELSE 
        EtatSuivant <= EcritureFiltre;
      END IF;
  END CASE;
END PROCESS NextStateProc;
\end{lstlisting}

Les processus \texttt{ClockedProc} et \texttt{NextStateProc} implémentent la machine séquentielle. \texttt{ClockedProc} met à jour l’état courant à chaque front montant de l’horloge et réinitialise l’état au démarrage. \texttt{NextStateProc} définit l’état suivant selon les conditions des signaux d’entrée et l’état courant, en suivant la logique de la machine de Mealy.

\subsubsection{Réseau Synchronisé de Sortie}

\begin{lstlisting}[style=VHDLStyle, caption={Reseau Synchronisé de Sortie}]
OutputProc_Synchro : PROCESS(H, nCLR)
BEGIN 
  IF (nCLR='0') THEN
    SelAdr <= (others => '0');
  ELSIF (H'EVENT AND H='1') THEN 
    CASE EtatCourant IS 
    WHEN EcritureFiltre =>
      IF (nCS_Synchro='1') THEN
        SelAdr <= D07_Synchro;
      END IF;
    WHEN OTHERS =>
    END CASE;
  END IF;
END PROCESS OutputProc_Synchro;
  
M_Received <= EtatLu(4);

\end{lstlisting}

Le processus \texttt{OutputProc\_Synchro} synchronise la sélection d’adresse \texttt{SelAdr} avec l’horloge \texttt{H}. Il est actif principalement pendant l’état \texttt{EcritureFiltre}, assurant que les données de l’entrée \texttt{D07\_Synchro} sont correctement mémorisées. Le signal \texttt{M\_Received} est également mis à jour pour refléter l’état du bit correspondant.

\section{Simulation des fonctions}

\subsection{Interface Microprocesseur}

Dans cette section, nous présentons la simulation du bloc \textit{Interface Microprocesseur} et l’analyse des chronogrammes obtenus.  
La simulation a été réalisée à l’aide d’un \textit{testbench}, implémenté sous la forme d’un bloc nommé \texttt{EnvTest\_InterfaceMicroprocesseur}, connecté au composant \texttt{InterfaceMicroprocesseur}.  
L’objectif est de vérifier la conformité du fonctionnement par rapport à l’automate décrit dans la section \textit{Architecture}.

\begin{figure}[H]
    \centering
    \includegraphics[width=0.95\linewidth]{images//Simulation/Chrono.png}
    \caption{Chronogramme de simulation de l’Interface Microprocesseur}
    \label{fig:placeholder}
\end{figure}

Le testbench (fourni en annexe) a pour rôle de reproduire l’environnement dans lequel le composant est amené à fonctionner.  
Il émule le comportement d’un microprocesseur en générant automatiquement les stimuli nécessaires à la validation du bloc testé.

\subsubsection{Déclarations et signaux}
Le testbench commence par la déclaration des librairies \texttt{IEEE}, nécessaires à la manipulation des types logiques et des vecteurs binaires.  
Les principaux signaux utilisés sont :
\begin{itemize}
  \item \texttt{CnD}, \texttt{RnW}, \texttt{nCS}, \texttt{nRST}, \texttt{H} : lignes de contrôle classiques d’une interface microprocesseur (commande/données, lecture/écriture, sélection du composant, reset, horloge),
  \item \texttt{OctetLu}, \texttt{EtatLu}, \texttt{SelAdr}, \texttt{D07} : bus de données et d’adresses sur 8 bits,
  \item \texttt{DecNbOctet}, \texttt{EtatLu\_RST}, \texttt{M\_Received}, \texttt{OctetLu\_RD} : signaux internes utilisés pour la communication avec le composant testé.
\end{itemize}

\subsubsection{Instanciation du composant testé}
Le composant \texttt{InterfaceMicroprocesseur} est instancié dans l’architecture de simulation.  
Il est relié à l’ensemble des signaux déclarés, permettant ainsi l’observation de son comportement face aux stimuli générés.

\subsubsection{Environnement de test}
Le composant \texttt{EnvTest\_InterfaceMicroprocesseur} simule le rôle du microprocesseur en générant automatiquement les signaux nécessaires :
\begin{itemize}
  \item génération de l’horloge (\texttt{H}),
  \item gestion du reset global (\texttt{nRST}),
  \item activation des commandes de lecture/écriture (\texttt{RnW}, \texttt{CnD}, \texttt{nCS}),
  \item pilotage du bus de données (\texttt{D07}).
\end{itemize}

Cet environnement est donné par plusieurs paramètres génériques :
\begin{itemize}
  \item \texttt{CLOCK\_PERIOD} : période d’horloge (50 ns),
  \item \texttt{RESET\_OFFSET} et \texttt{RESET\_DURATION} : moment et durée du reset (500 ns et 300 ns),
  \item \texttt{ACCESS\_TIME} et \texttt{HOLD\_TIME} : contraintes temporelles d’accès et de maintien (40 ns et 70 ns).
\end{itemize}

\begin{figure}[H]
    \centering
    \includegraphics[width=0.95\linewidth]{images/Simulation/banc_test.png}
    \caption{Block Diagramme de test de l’Interface Microprocesseur}
    \label{fig:placeholder}
\end{figure}

\subsubsection{Stimuli supplémentaires}
Un processus spécifique (\texttt{StimProc}) complète la génération des signaux.  
Après la fin du reset, il impose des valeurs constantes sur certaines lignes :
\begin{itemize}
  \item \texttt{OctetLu} $\leftarrow$ 10 (codé sur 8 bits),
  \item \texttt{EtatLu} $\leftarrow$ 8 (codé sur 8 bits).
\end{itemize}
Ces valeurs permettent de vérifier la gestion correcte des données reçues par l’interface.  
La simulation est ensuite maintenue en attente infinie.

\subsubsection{Analyse du chronogramme de simulation}
L’analyse du chronogramme met en évidence le comportement attendu du composant :
\begin{itemize}
  \item lorsque les signaux de contrôle actifs à l’état bas (\texttt{nRST}, \texttt{nCS}) sont à l’état haut, aucune action n’est effectuée,
  \item lorsque ces signaux sont activés (passage à l’état bas), le composant réagit conformément à l’automate interne,
  \item les signaux \texttt{RnW} et \texttt{CnD} permettent de sélectionner respectivement les opérations de lecture/écriture et le type d’accès (commande ou données),
  \item les valeurs imposées sur \texttt{OctetLu} et \texttt{EtatLu} sont correctement lues via le bus de données \texttt{D07}.
\end{itemize}

Ce chronogramme confirme ainsi le bon fonctionnement du composant \texttt{InterfaceMicroprocesseur} :  
après la levée du reset, l’environnement de test génère des cycles de lecture et d’écriture auxquels le composant répond correctement, en échangeant les données prévues et en activant les signaux de contrôle appropriés.


\section{Synthèse des fonctions}

Une fois la validation en simulation des différents blocs effectuée, il est nécessaire de réaliser la synthèse sur FPGA afin d’observer les ressources logiques attribuées à notre système. 
Dans le cadre de ce projet, nous avons utilisé un FPGA \textit{AMD Xilinx Artix-7}, plus précisément le modèle \texttt{7A35TCPG236}. 
L’objectif de cette étape est d’analyser les ressources logiques mobilisées, ainsi que les éléments matériels effectivement utilisés par notre conception.
\newline

\subsection{Interface Microprocesseur}

Le schéma RTL (\textit{Register Transfer Level}) représente une implémentation 
synthétisée d’un module matériel décrit en VHDL ou Verilog. 
Il illustre les registres, les multiplexeurs, les portes logiques, 
ainsi que la logique séquentielle et combinatoire du circuit.


\begin{figure}[H]
    \centering
    \includegraphics[width=0.9\linewidth]{images/Synthe/RTL_Shematic.png}
    \caption{Schéma RTL InterfaceMicroprocesseur}
    \label{fig:placeholder}
\end{figure}

\subsection*{Structure générale}
Le schéma peut être décomposé en plusieurs parties :
\begin{itemize}
    \item \textbf{Entrées principales} : signaux tels que 
    \texttt{H}, \texttt{CnD}, \texttt{RnW}, \texttt{nRST}, \texttt{nCS}, etc.
    \item \textbf{Registres (Flip-Flops D)} : éléments synchronisés par l’horloge, 
    servant à mémoriser l’état interne du circuit.
    \item \textbf{Multiplexeurs (MUX)} : permettent de sélectionner une donnée 
    parmi plusieurs, selon les conditions de contrôle.
    \item \textbf{Logique combinatoire} : réalisée par des portes AND, OR, NOT 
    et XOR, afin de générer les conditions de transition et les sorties.
    \item \textbf{Sorties} : plusieurs signaux dérivés de l’état interne, 
    comme \texttt{State\_XX}, \texttt{Output\_XX}, etc.
\end{itemize}

\subsection*{Fonctionnement global}
Le circuit implémente une \textbf{machine à états finis} :
\begin{itemize}
    \item Les \textbf{registres} contiennent l’état courant.
    \item La \textbf{logique combinatoire} calcule l’état suivant 
    en fonction de l’état courant et des entrées.
    \item Les \textbf{multiplexeurs} dirigent les transitions entre états.
    \item Les \textbf{sorties} sont activées ou désactivées selon l’état courant 
    et certaines combinaisons d’entrées.
\end{itemize}

\begin{figure}[H]
    \centering
    \includegraphics[width=0.9\linewidth]{images/Synthe/mux_tris.png}
    \caption{Partie opérative avec multiplexeur et Tristate : InterfaceMicroprocesseur}
    \label{fig:placeholder}
\end{figure}

De plus, nous pouvons retrouver la partie opérative dessinée en classe, lors de nos TD, qui permet, 
grâce à un multiplexeur, de sélectionner \textbf{EtalLu} ou \textbf{OctetLu} pour l'envoyer vers 
une porte \textbf{Tristate}.  
Cela démontre la cohérence entre la réalisation théorique et la mise en œuvre pratique.
\newline

À la suite de la synthèse logique nous pouvons avoir la la synthèse matériel qui transforme la 
logique en ressource matérielle.

\begin{figure}[H]
    \centering
    \includegraphics[width=0.9\linewidth]{images/Synthe/RTL_HARD_HDL.jpg}
    \caption{Synthese matérielle InterfaceMicroprocesseur}
    \label{fig:placeholder}
\end{figure}

Cette transformation consiste à mapper les éléments logiques du schéma RTL, tels que les portes 
\textbf{ AND, OR, NOT} ou les \textbf{multiplexeurs}, sur les \textbf{ressources matérielles 
physiques} disponibles dans le FPGA, notamment :  
\newline

\begin{itemize}
    \item \textbf{LUT (Look-Up Tables)} : les fonctions combinatoires, comme les portes logiques ou les multiplexeurs, sont réalisées à l’aide de LUT. Chaque LUT peut implémenter n’importe quelle fonction booléenne sur un nombre limité d’entrées, ce qui permet de reproduire fidèlement la logique définie dans le HDL.
    \item \textbf{Flip-flops} : les éléments séquentiels tels que les registres ou les bascules sont mappés sur des flip-flops pour stocker les bits et synchroniser les signaux dans le temps.
    \item \textbf{Buffers} : certains chemins logiques nécessitent des buffers pour renforcer les signaux ou adapter les niveaux électriques, garantissant ainsi la stabilité et l’intégrité du routage sur la puce.
\end{itemize}

Grâce à cette conversion, le design passe d’une \textbf{représentation abstraite de la logique} à une \textbf{implémentation matérielle concrète}, optimisée pour le FPGA cible. Cela permet non seulement de visualiser l’organisation physique des composants.
\newline

\section{Routages des Fonctions
}
\subsection{Interface Microprocesseur}

Après l'étape de \textbf{synthèse}, nous pouvons nous intéresser à l'assignation des ressources 
matérielles du système.\\
Pour cela, nous utilisons le logiciel \textbf{Vivado} (étant donné que les outils de HDL n'étaient 
pas disponibles le jour du TP), qui permet de générer un schéma RTL ainsi qu'une vue du routage 
associé à l'interface microprocesseur.
\newline

Dans un premier temps, Nous allons resynthétiser le design afin d'obtenir le schéma RTL.
Ce schéma, présenté ci-dessous, illustre les ressources logiques utilisées pour implémenter 
l'interface microprocesseur.
\newline

\resizebox{\textwidth}{!}{%
    \includegraphics[angle=-90]{images/Routage/pdf_recadre_recadre.pdf}%
}

\vspace{10pt}

Par la suite, nous pouvons également observer la synthèse matérielle réalisée sur Vivado, qui 
transforme les ressources logiques en ressources matérielles pour le FPGA.
\newline

\begin{figure}[H]
    \centering
    \includegraphics[width=0.8\linewidth]{images/Routage/schematic_RTL_VIVADO_recadre.pdf}
    \caption{Synthèse matérielle Vivado}
    \label{fig:rout_general}
\end{figure}

Cette synthèse nous montre que la logique est transformée en ressource matérielle, notamment des 
LUT (Look-Up Tables) et des Flip-Flops, qui sont les éléments de base pour implémenter la logique 
dans un FPGA.
\newline

Les \textbf{LUT} (Look-Up Tables), éléments fondamentaux d'un FPGA, peuvent être considérées comme 
des portes logiques programmables capables de réaliser toute fonction combinatoire. 
Elles constituent la base de l'implémentation matérielle et offrent une vision schématique complète 
du système.  
\newline

Prenons l'exemple d'une \textbf{LUT3} :  

\begin{figure}[H]
    \centering
    % Première image
    \begin{subfigure}[b]{0.55\linewidth}
        \centering
        \includegraphics[width=\linewidth]{images/Routage/LUT_EXEMPLE.png}
        \caption{Exemple de LUT}
        \label{fig:lut_exemple}
    \end{subfigure}
    \hfill
    % Deuxième image
    \begin{subfigure}[b]{0.25\linewidth}
        \centering
        \includegraphics[width=\linewidth]{images/Routage/TABLE_LUT.png}
        \caption{Table de vérité associée}
        \label{fig:table_lut}
    \end{subfigure}
    \caption{Illustration d'une LUT et de sa table de vérité}
    \label{fig:lut_complete}
\end{figure}

Grace à la table de vérité de la LUT nous remarquons que cette LUT3 implémente une fonction logique ET à trois entrées.
\newline

\begin{figure}[H]
    \centering
    \includegraphics[width=0.8\linewidth]{images/Routage/Mux_Tris_Vivado.png}
    \caption{Synthèse Logique Vivado}
    \label{fig:rout_general}
\end{figure}

Nous observons également la présence de la partie opérative de l'interface Microprocesseur.
\newline

\medskip

La suite du flot de conception consiste à lancer l'\textbf{implémentation} du système afin d'obtenir le routage complet. 
Vivado propose alors une vue générale du FPGA, mettant en évidence ses différentes zones fonctionnelles :  

\begin{figure}[H]
    \centering
    \includegraphics[width=0.8\linewidth]{images/Routage/Rout_1.png}
    \caption{Slice du FPGA après routage}
    \label{fig:rout_general}
\end{figure}

En effectuant un zoom, il est possible de constater que le système a été implémenté dans la zone \textbf{0} du FPGA. 
Nous observons que la zone située à gauche correspond aux entrées de chaque variable, 
celles-ci étant toutes reliées à des \textbf{buffers}.

\begin{figure}[H]
    \centering
    % Première image (anciennement deuxième)
    \begin{subfigure}[b]{0.45\linewidth}
        \centering
        \includegraphics[width=\linewidth]{images/Routage/Rout_2.png}
        \caption{Localisation du système dans la zone 0 du FPGA}
        \label{fig:rout_zone0}
    \end{subfigure}
    \hfill
    % Deuxième image (anciennement première)
    \begin{subfigure}[b]{0.50\linewidth}
        \centering
        \includegraphics[width=\linewidth]{images/Routage/buff.png}
        \caption{Buffer associé à nCS}
        \label{fig:buff_ncs}
    \end{subfigure}
    \caption{Vue du système routé et des buffers associés sur le FPGA}
    \label{fig:rout_buff}
\end{figure}



Un zoom encore plus détaillé permet d'observer le câblage interne des ressources identifiées lors de la synthèse.  
Par exemple, une \textbf{LUT3} est câblée de la manière suivante :  

\begin{figure}[H]
    \centering
    % Première image
    \begin{subfigure}[b]{0.35\linewidth}
        \centering
        \includegraphics[width=\linewidth]{images/Routage/Rou_3.png}
        \caption{Connexion interne d'une LUT3}
        \label{fig:rout_lut3a}
    \end{subfigure}
    \hfill
    % Deuxième image
    \begin{subfigure}[b]{0.55\linewidth}
        \centering
        \includegraphics[width=\linewidth]{images/Routage/Rou_4.png}
        \caption{Schéma détaillé du câblage LUT3}
        \label{fig:rout_lut3b}
    \end{subfigure}
    \caption{Exemple de routage d'une LUT3 dans le FPGA}
    \label{fig:rout_lut3}
\end{figure}

\medskip

Le résultat final est une représentation complète et hiérarchisée du système, directement mappée sur le FPGA. 
\textbf{Vivado} offre ainsi la possibilité de visualiser l'ensemble du flot, depuis la description logique RTL jusqu'au routage physique détaillé.  

En résumé, la conception suit une progression en trois étapes :  
\begin{itemize}
    \item la \textbf{synthèse} génère une description logique optimisée du système (LUT, registres, blocs fonctionnels) ;  
    \item le \textbf{placement} attribue ces ressources aux cellules physiques du FPGA ;  
    \item le \textbf{routage} établit les interconnexions nécessaires au bon fonctionnement du circuit.  
\end{itemize}

Cette approche permet de passer d'une description abstraite en langage HDL à une implémentation matérielle 
concrète, où chaque fonction logique est traduite en ressources physiques. 
Vivado fournit alors une vision globale et détaillée du FPGA, allant de la logique combinatoire jusqu'au câblage interne des composants.


\section{Conclusion}

Pour ce premier rapport concernant le projet \textbf{Réception LIN} de conception de circuit numérique, 
nous avons suivi plusieurs étapes successives afin de concevoir ce système.  

Dans un premier temps, lors des séances de TD, nous avons décomposé notre étude en plusieurs parties afin de répondre au cahier des charges. 
L’utilisation du \textbf{diagramme en Y} nous a permis d’analyser séparément chacune de ces étapes et de structurer notre démarche. 
\newline

La phase de \textbf{spécification fonctionnelle} nous a permis de définir l’ensemble des ressources 
nécessaires, 
en lien direct avec le cahier des charges, ainsi que le nombre de blocs et les signaux de base.  
\newline

Ensuite, la \textbf{solution architecturale} a permis de préciser chaque partie en les reliant à des 
modèles connus 
(machines séquentielles, machines de Moore ou machines de Mealy). 
Cette étape a été essentielle pour découper notre système en une partie opérative et une partie commande :  
\newline

\begin{itemize}
    \item la partie opérative a été conçue à partir de blocs logiques simples (multiplexeurs, bascules D, etc.) ;  
    \item la partie commande a été décrite sous forme d’automates, associés à des machines de Moore ou de Mealy, 
    afin d’obtenir une description claire et structurée.  
\end{itemize}
Cette méthodologie nous a permis d’écrire un code plus simple, lisible et cohérent.  

La phase de \textbf{simulation} a ensuite validé le fonctionnement du système en stimulant les différents ports d’entrée 
et en observant les sorties.  
\newline

La phase de \textbf{synthèse} nous a permis de vérifier la logique interne du système à travers les schémas RTL générés. 
Ces schémas ont ensuite été exploités dans Vivado, qui a associé les différentes fonctions logiques aux \textbf{LUT}.  
\newline

Enfin, l’\textbf{implémentation} nous a donné accès au routage sur FPGA. 
Cette étape a permis de vérifier concrètement l’affectation des ressources matérielles et le câblage interne du système.  
\newline

En conclusion, ce premier travail nous a permis de valider l’\textbf{interface microprocesseur}. 
La suite du projet consistera à finaliser la conception complète du récepteur LIN afin de répondre à l’intégralité du cahier des charges.


\section{Annexes}

\subsection{Testbench InterfaceMicroprocesseur}

\begin{lstlisting}[style=VHDLStyle, caption={Testbench InterfaceMicroprocesseur}]
LIBRARY ieee;
USE ieee.std_logic_1164.all;
USE ieee.std_logic_arith.all;

ENTITY EnvTest_InterfaceMicroprocesseur IS
   GENERIC( 
      CLOCK_PERIOD   : time := 50 ns;
      RESET_OFFSET   : time := 500 ns;
      RESET_DURATION : time := 300 ns;
      ACCESS_TIME    : time := 40 ns;
      HOLD_TIME      : time := 70 ns
   );
   PORT( 
      M_Received : IN     std_logic;
      CnD        : OUT    std_logic;
      H          : OUT    std_logic;
      RnW        : OUT    std_logic;
      nCS        : OUT    std_logic;
      nRST       : OUT    std_logic;
      D07        : INOUT  std_logic_vector (7 DOWNTO 0)
   );
-- Declarations

END EnvTest_InterfaceMicroprocesseur ;

--
ARCHITECTURE arch OF EnvTest_InterfaceMicroprocesseur IS
TYPE DefState IS (Waiting, DataReading, StateReading, FilterWriting);

SIGNAL ProcessorState : DefState;

BEGIN
  
ClockGeneratorProc : PROCESS
BEGIN
  H <= '0';
  WAIT FOR CLOCK_PERIOD/2;
  H <= '1';
  WAIT FOR CLOCK_PERIOD/2;
END PROCESS ClockGeneratorProc;

ResetGeneratorProc : PROCESS
BEGIN
  nRST <= '1';
  WAIT FOR RESET_OFFSET;
  nRST <= '0';
  WAIT FOR RESET_DURATION;
  nRST <= '1';
  WAIT;
END PROCESS ResetGeneratorProc;

ProcessorBehaviorProc : PROCESS
BEGIN
  D07 <= (others => 'Z');
--Waiting cycle--
  ProcessorState <= Waiting;
  nCS <= '1';
  CnD <= '1';
  RnW <= '1';
  WAIT FOR RESET_OFFSET+RESET_DURATION+2*CLOCK_PERIOD;    
--Reading data cycle--
  ProcessorState <= DataReading;
  WAIT FOR ACCESS_TIME;
  nCS <= '0';
  CnD <= '0';
  RnW <= '1';
  WAIT FOR 2*CLOCK_PERIOD;
--Waiting cycle--
  ProcessorState <= Waiting;
  nCS <= '1'; 
  CnD <= '1';
  RnW <= '1';
  WAIT FOR 2*CLOCK_PERIOD-ACCESS_TIME;
--Reading state cycle--
  ProcessorState <= StateReading;
  WAIT FOR ACCESS_TIME;
  nCS <= '0';
  CnD <= '1';
  RnW <= '1';
  WAIT FOR 2*CLOCK_PERIOD;
--Waiting cycle--
  ProcessorState <= Waiting;
  nCS <= '1';
  CnD <= '1';
  RnW <= '1';
  WAIT FOR 2*CLOCK_PERIOD-ACCESS_TIME;
--Writing cycle--
  ProcessorState <= FilterWriting;
  WAIT FOR ACCESS_TIME;
  nCS <= '0';
  CnD <= '0';
  RnW <= '0';
  D07 <= (others => '1');
  WAIT FOR 2*CLOCK_PERIOD;
--Waiting cycle--
  ProcessorState <= Waiting;
  nCS <= '1';
  CnD <= '1';
  RnW <= '1';
  WAIT FOR HOLD_TIME;
  D07 <= (others => 'Z');
  WAIT;
END PROCESS ProcessorBehaviorProc;

END ARCHITECTURE arch;

\end{lstlisting}

\listoffigures

\end{document}